\documentclass{beamer}
\usepackage[english,french]{layout}
\usepackage[utf8]{inputenc}
\usepackage[english]{babel}
\usepackage[T1]{fontenc}
\usepackage{lmodern}
\usepackage{textcomp}
\usepackage{amsmath, soul, color, multicol, type1cm, verbatim, latexsym, dsfont, float, listings,alltt}
\usepackage[official]{eurosym}
\usepackage{beamerthemesplit}
\usetheme{Frankfurt}
\usefonttheme{professionalfonts}
\setbeamercovered{transparent}
%NeSI Colors <-------------------------------------------------------------------------------------
\usecolortheme{lily}
\usecolortheme[RGB={47, 68, 71}]{structure} 
\definecolor{nesidark}{HTML}{2F4447}
\definecolor{nesilight}{HTML}{CED9DF}
\definecolor{nesigrey}{gray}{0.7}
\definecolor{nesilightgrey}{gray}{0.98}
\definecolor{nesidarkgrey}{gray}{0.3}
\definecolor{nesiblue}{HTML}{2B9FC2}
\setbeamercolor{block title}{fg=black,bg=nesigrey}
\setbeamercolor{block body}{bg=nesilightgrey,fg=nesidarkgrey}
\setbeamercolor{block body alerted}{bg=white,fg=black}
\setbeamercolor{alerted text}{bg=white,fg=black}
%NeSI Title <---------------------------------------------------------------------------------------
\setbeamerfont{title}{size=\huge}
\frenchspacing
\hyphenation{NeSI}
%NeSI Template parameters <-------------------------------------------------------------------------
\setbeamertemplate{blocks}[default]
\useinnertheme{circles}
\setbeamertemplate{title page}[default][center,rounded=false,shadow=false]
\newcommand\BackgroundPicture[1]{%
\setbeamertemplate{background}{%
\parbox[c][\paperheight]{\paperwidth}{%
\vfill \hfill \includegraphics[height=0.9\paperheight]{#1}
\hfill \vfill
}}}

%Content Starts Here <-------------------------------------------------------------------------------
\title{Enabling High--Throughput Research on HPC Systems}
%\subtitle{Computational Science team}
\author{Dr Fran{\c c}ois Bissey \\ (francois.bissey@canterbury.ac.nz)}
\date{}

\begin{document}

{
\setbeamertemplate{background canvas}{\includegraphics[height=0.99\paperheight]{NeSI_img/Slide00.png}} 
\begin{frame}[plain]
\vspace{1cm}
\titlepage
\end{frame}
}


% This will generate the outline. If you have several topics, uncomment the multicols 
\begin{frame}
\frametitle{Outline}
% \begin{multicols}{2} 
   \tableofcontents
% \end{multicols}
\end{frame}


%%%%%%%%%%%%%%%%%%%%%%%%%%%%%%%%%%%%%%%%%%%%%%%%%%%%%%%%%%%%%%%%%%%%%%%%%%%%%%%%%%%%%%%%%%%%%%%
%%%%%%%%%%%%%%%%%%%%%%%%%%%%%%%%%%%%%%%% Some Examples %%%%%%%%%%%%%%%%%%%%%%%%%%%%%%%%%%%%%%%%%%%
%%%%%%%%%%%%%%%%%%%%%%%%%%%%%%%%%%%%%%%%%%%%%%%%%%%%%%%%%%%%%%%%%%%%%%%%%%%%%%%%%%%%%%%%%%%%%%%

\section{Introduction}

\frame[t]
{
\frametitle{Introduction}
\begin{block}{Two types of fundamental workflow for HPC}
\begin{itemize}
 \item Large distributed memory jobs (MPI)
 \item High-throughput jobs
\end{itemize}
\end{block}

\begin{block}{High-throughput jobs covers}
\begin{itemize}
 \item Monte-Carlo simulations
 \item Parameter space sweeps
 \item Processing of large data sets
\end{itemize}
It is characterised by doing a task over and over again with a different input.
\end{block}
}

\section{The Problem}

\frame[t]
{
\frametitle{The problem}
\begin{block}{MPI machines}
Machine like the Power7 cluster at BlueFern and Fitzroy at NIWA are geared towards distributed memory jobs (MPI).
\end{block}

\begin{block}{High-throughput job}
Running high-throughput jobs efficiently on these systems requires a little bit of thinking ahead and care.
We will "steal" the MPI workflow to achieve this. Machine like Pan in Auckland is geared more towards high-throughput jobs but applying the technics presented could be beneficial on it as well.
\end{block}
}

\section{Technical Overview}
\frame[t]
{
\frametitle{Technical Overview}
 There are three structured approaches to run our jobs on a MPI machine
\begin{enumerate}
 \item MPMD
 \item batcher
 \item custom (usually a form of the previous two but integrated within your workflow)
\end{enumerate}
We will use three case studies to show the problems and the solutions adopted.
}

\section{Case Studies}

\subsection{presentation}
\frame[t]{
\frametitle{Case Studies}
\begin{block}{Presentation}
We'll go through the work of three researchers showing various problems and solutions and we'll summarize their pros and cons.
\begin{enumerate}
 \item Ehsan's simulations (the flood)
 \item Nick's network (MPMD in testing)
 \item Jing Wang: DNA matching (batcher)
\end{enumerate}
\end{block}
}
\subsection{Eshan's Simulations}
\frame[t]{
\frametitle{Ehsan's Simulations}
Ehsan is a PhD student in computer science. His problem is simulating wireless networks to improve the efficiency of the communication in mobile devices.

\begin{itemize}
\item first submission to the cluster was 40,000 simulations
\item done 4 at a time
\item first reaction: increase the number of simulations he can run (after all they are not big MPI jobs)
\item Bad idea: the simulations were spread all over the cluster blocking big MPI jobs
\item developed a custom solution to contain him to a set of nodes
\item we later found out that it was equivalent to using MPMD
\end{itemize}
We now identify high-throughput project quickly and offer them better solutions as soon as possible.
}

\subsection{Nick's networks}
\frame[t]{
\frametitle{Nick's Networks}
\begin{itemize}
\item Nick Baker is a PhD student in biological sciences with a background in mathematics. He studies ecological networks (complex food chains).

\item His work focuses on analysing networks to find "keystone" species that are essential to the network. The work is what we call embarassingly parallel, he has to go through a high number of networks and perfoorm his analyses on each and everyone of them.

\item I spotted him on the cluster running running four jobs at a time out of close to a hundred. There was definitely room for improvement. He became a test subject for MPMD
\end{itemize}
}
\frame[t]{
\frametitle{MPMP: Multiple Program Multiple Data}
\begin{multicols}{2}
\begin{figure}
\includegraphics[scale=0.5]{../smpd_task.pdf}
\end{figure}
SMPD: Instances of a program process the data distributed amongst them.
\begin{figure}
\includegraphics[scale=0.5]{../mpmd_task.pdf} 
\end{figure}
MPMD: Programs, not necessarily identical, process distinct data.
\end{multicols}
}
{
\setbeamertemplate{background canvas}{\includegraphics[height=0.99\paperheight]{NeSI_img/Slide00.png}} 
\begin{frame}[plain]
\begin{center}
{\Huge Questions \& Answers}
\end{center}
\end{frame}
}


\end{document} 
